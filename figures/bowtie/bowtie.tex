% Bowtie Diagram 
% Author: 3implieschaos
\documentclass[border=10pt]{standalone}
\usepackage{tikz}
\usepackage{pgfplots}
\pgfplotsset{compat=1.16}
\usetikzlibrary{positioning}
\usetikzlibrary{shapes.multipart}
\usetikzlibrary{patterns}


\begin{document}
  \def\threatsx{-16}
  \def\consequencesx{16}
  \begin{tikzpicture}[
      auto,
      node distance=0cm,
      every text node part/.style={align=center},
      topevent/.style={circle,
                      draw,
                      very thick,
                      fill=red,
                      font=\sffamily\Large\bfseries,
                      inner sep=5pt,
                      minimum size=30mm,
                      align=center},
      escalation/.style={rectangle,
                  draw,
                  very thick,
                  fill=yellow,
                  font=\sffamily\Large\bfseries,
                  inner sep=5pt,
                  minimum width=45mm,
                  minimum height=10mm},
      threat/.style={rectangle,
                  draw,
                  very thick,
                  fill=blue,
                  text=white,
                  font=\sffamily\Large\bfseries,
                  inner sep=5pt,
                  minimum width=45mm,
                  minimum height=10mm},
      consequence/.style={rectangle,
                  draw,
                  very thick,
                  fill=red,
                  font=\sffamily\Large\bfseries,
                  inner sep=5pt,
                  minimum width=45mm,
                  minimum height=10mm},
      control/.style={rectangle,
                    draw=black!50,
                    fill=black!20,
                    thick,
                    minimum width=0.8cm,
                    minimum height=1.8cm},
      paths/.style={-, ultra thick}
  ]

  % see https://tex.stackexchange.com/questions/29359/pgfplots-how-to-fill-the-area-under-a-curve-with-oblique-lines-hatching-as-a/29367#29367
  \tikzset{
        hatch distance/.store in=\hatchdistance,
        hatch distance=10pt,
        hatch thickness/.store in=\hatchthickness,
        hatch thickness=2pt
    }

    \makeatletter
    \pgfdeclarepatternformonly[\hatchdistance,\hatchthickness]{flexible hatch}
    {\pgfqpoint{0pt}{0pt}}
    {\pgfqpoint{\hatchdistance}{\hatchdistance}}
    {\pgfpoint{\hatchdistance-1pt}{\hatchdistance-1pt}}%
    {
        \pgfsetfillcolor{yellow}
        \pgfsetcolor{black}%{\tikz@pattern@color}
        \pgfsetlinewidth{\hatchthickness}
        \pgfpathmoveto{\pgfqpoint{0pt}{0pt}}
        \pgfpathlineto{\pgfqpoint{\hatchdistance}{\hatchdistance}}
        \pgfusepath{stroke}
    }
    \makeatother


  % Hazard
  \draw[yellow, fill=yellow, opacity=0.5, pattern=flexible hatch] (-2,5) rectangle (2,3);
  \node[escalation] (hazard) at (0,4) {Laptop Contains \\ Financial Data};

  % Top Event
  \node [topevent] (top_event) at (0,0) {Loss of Laptop};
  \draw [paths] (hazard) to node { } (top_event);

  % Threats
  \node [threat] (threat1) at (\threatsx, 4) {Theft};
  \node[control] (controlt1) at (\threatsx/2, 4/2) {};
  \node [below=of controlt1, font=\sffamily\Large\bfseries] {Security\\Locks};
  \draw [paths] (threat1) to node { } (controlt1);
  \draw [paths] (controlt1) to node { } (top_event);

  \node [threat] (threat2) at (\threatsx, -4) {Misplaced};
  \node[control] (controlt2) at (\threatsx/2, -4/2) {};
  \node [below=of controlt2, font=\sffamily\Large\bfseries] {Geolocation};
  \draw [paths] (threat2) to node { } (controlt2);
  \draw [paths] (controlt2) to node { } (top_event);


  % Consequences
  \node [consequence] (consequence1) at (\consequencesx, 7) {Financial Loss};
  \draw [paths] (top_event) to node { } (consequence1);


  \node [consequence] (consequence2) at (\consequencesx, 0) {Data Loss};
  \node[control] (control2) at (\consequencesx/2, 0) {};
  \node [below=of control2, font=\sffamily\Large\bfseries] {Encryption};
  \draw [paths] (top_event) to node { } (control2);
  \draw [paths] (control2) to node { } (consequence2);


  \node [consequence] (consequence3) at (\consequencesx, -7) {Financial Compromise};
  \node[control] (control3a) at (\consequencesx-10, -4) {};
  \node [below=of control3a, font=\sffamily\Large\bfseries] {Encryption};
  \draw [paths] (top_event) to node { } (control3a);
  \node[control] (control3b) at (\consequencesx-5, -6) {};
  \node [below=of control3b, font=\sffamily\Large\bfseries] {MFA};
  \draw [paths] (control3a) to node { } (control3b);
  \draw [paths] (control3b) to node { } (consequence3);

  \node [escalation] (escalation1) at (\consequencesx-2, -2.5) {User disables \\ Encryption};
  \node[control] (controle1) at (\consequencesx-5, -2.5) {};
  \node [below=of controle1, font=\sffamily\Large\bfseries] {Policy \\ requires};
  \draw [paths] (control2) to node { } (controle1);
  \draw [paths] (control3a) to node { } (controle1);
  \draw [paths] (controle1) to node { } (escalation1);


  \end{tikzpicture}

\end{document}
